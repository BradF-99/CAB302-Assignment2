% CAB302 Assignment 2 Report 
% Authors: Alex Butler, Brad Fuller & Joshua O'Riordan
% LaTeX assistance: Vincent Nguyen
% Version 1.2 2019-05-29

% NOTE: May require building twice in order for Contents page to show correctly.

\documentclass[12pt]{article} % This states what size and type of document. Generally leave this as is.

\usepackage{fullpage} % Essential - makes you actually use the whole
\usepackage{amsmath} % Essential - for math
\usepackage{amssymb} % Essential - for math
\usepackage{amstext} % Essential - for math
\usepackage{amsfonts} % Essential - for math
\usepackage{amsthm} % Essential - for math
\usepackage{graphicx} % Essential - for inserting images
\usepackage{hyphenat} % I like this - removes hyphenated continues on new lines
\usepackage{cancel} % I like this - allows use of \cancel{something}
\graphicspath{ {images/} } % I like this - a different way for inserting images
\usepackage{parskip} % I like this - removes first line indentation
\usepackage{microtype} % Added for underfull fixing
% Packages that I use.


\begin{document} % Everything begins and ends with this

\title{Development Report and Documentation
		 \\ \large CAB302 - Assignment 2}

\author {Alex Butler,  Bradley Fuller \& Joshua O'Riordan}
\maketitle

\newpage

\tableofcontents

\newpage

\section{Statement of Completeness}

The solution includes the following features: 
\begin{center}
	\begin{tabular}{|c|c|}
		\hline  \textbf{Drawing - Line} & Complete - Tested Working \\
		\hline  \textbf{Drawing - Plot} & Complete - Tested Working \\
		\hline  \textbf{Drawing - Rectangle} & Complete - Tested Working \\
		\hline  \textbf{Drawing - Ellipse} & Complete - Tested Working \\
		\hline  \textbf{Drawing - Polygon} & Complete - Tested Working \\
		\hline  \textbf{Drawing - Pen Colour} & Complete - Tested Working \\
		\hline  \textbf{Drawing - Fill Colour} & Complete - Tested Working \\
		\hline  \textbf{File Handling - Read} & Complete - Tested Working \\
		\hline  \textbf{File Handling - Write} & Still to be implemented \\
		\hline  \textbf{GUI - Layout} & Complete - Tested Working \\
		\hline  \textbf{GUI - Buttons} & Complete - Tested Working \\
		\hline  \textbf{GUI - Menu Bar} & Complete - Tested Working \\
		\hline  \textbf{GUI - Colour Picker} & Complete - Tested Working \\
		\hline  \textbf{Additional - Undo History} & Still to be implemented \\
		\hline  \textbf{Additional - Bitmap Export} & Still to be implemented \\
		\hline
	\end{tabular}
\end{center}
By the due date, we were able to complete all of the essential requirements and the required amount of two additional requirements - being Undo History, and Bitmap Export. 

\newpage

\section{Statement of Contribution}

{\large Alex Butler}\\
Alex implemented all of the Shape classes, as well as the Shape interface and enumerator. He also wrote the unit tests for testing each Shape class, and the first iteration of the GUI (mainly for testing purposes). 
\\\\{\large Bradley Fuller}\\
Bradley implemented the file handling and validation classes, as well as the unit testing for said classes. He also wrote and configured the Apache Ant build script, and the TravisCI configuration for continuous integration and delivery to GitHub Releases.
\\\\{\large Joshua O'Riordan}\\
Joshua implemented the final iteration of the GUI, including the general layout, buttons, menu bar and colour picker. He also wrote the additional requirements functionality (being Undo History and Bitmap Export), as well as their respective unit tests.

\newpage

\section{Software Development Practices}

The team utilised Agile's philosophy of iterative and incremental changes, through the use of continuous integration and deployment. This was achieved through the use of TravisCI and GitHub Releases. Whenever a change was made in any branch of the repository, TravisCI would create a temporary Docker container with Ubuntu 16.04 and Oracle JDK 11. All dependencies are then downloaded - in our case, JUnit 5.5.0 and JavaFX, as well as Apache Ant 1.10.6. Ant is used for creating all build folders, compiling the solution, and then running all unit tests (which are stored in the src/tests/ directory). We would then receive a report (via email) of the testing and if any errors occurred. However, if the change was made in the "master" branch of the repository, TravisCI would run Ant's "dist" script, which packages the compiled Java classes in to a JAR file. This JAR file would then be given a version number (which during development was TravisCI's build number) and uploaded to GitHub Releases. Pull requests would also be tested against the "master" branch of the repository to ensure there would be no breaking changes.

Due to other units and other assessment deadlines, we were not able to follow Agile's sprint methodology and strict timing requirements exactly. However, each feature  of the solution was completed in sprints, each being in their own branch in the repository. Once these were complete, the branch would be merged in to master, and work on the next feature could begin. The only exception to this was the File Handling features, as they required the GUI to be fully implemented before they could be merged - however they were tested thoroughly beforehand without the use of a GUI, so the integration was very simple. We also had a kanban board of features that were to be implemented in the project, as well as an automated kanban board for bug triage and fixing.

Test-driven development was also used where possible, especially in the file read and write classes of the solution. The file reading class had over 20 tests of it's own, to make sure that exceptions were being thrown when they should have. Different test cases were used such as invalid commands, invalid co-ordinates, invalid colours, non-VEC file tests, incorrect file types etc.


\newpage

\section{Software Architecture}

As with every Java project, there is a Main class - in this instance, the Main class calls the MainScreen class in order to initialise the GUI. When initialised, the GUI class instantiates the ComponentsClass constructor, which in turn instantiates every Shape class and prepares them for drawing. As well as this, the ShapesEnum enumerator is constructed which includes a list of all shapes. This enumerator is used to aid in drawing shapes for the GUI. The FileRead and FileWrite classes are also instantiated as part of the GUI class - this is in order to read to and write from the Components constructor.

\newpage

\section{Usage of Advanced Object-Oriented Principles}
{\large Abstraction}\\
Some details about abstraction usage.
\\\\{\large Encapsulation}\\
Some details about encapsulation usage.
\\\\{\large Inheritance}\\
Some details about inheritance usage.
\\\\{\large Polymorphism}\\
Some details about polymorphism usage.

\newpage

\section{Using the Software}

Add documentation for using the software (presumably documentation written for the
\\ designers). This needs to be comprehensive in describing all facets of the software.

\end{document}
